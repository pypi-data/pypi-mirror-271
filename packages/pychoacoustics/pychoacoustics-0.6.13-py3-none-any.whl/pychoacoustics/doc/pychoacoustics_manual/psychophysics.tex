\chapter{Psychophysics}
\section{Available Paradigms}
\label{sec:paradigms}
\subsubsection{Adaptive}
This paradigm implements the ``up/down'' adaptive procedures described by \citeA{Levitt1971}. It can be used with $n$-intervals, $n$-alternatives forced choice tasks, in which $n-1$ ``standard'' stimuli and a single ``comparison'' stimulus are presented each in a different temporal interval. The order of the temporal intervals is randomized from trial to trial. The ``comparison'' stimulus usually differs from the ``standard'' stimuli for a single characteristic (e.g.\ pitch or loudness), and the listener has to tell in which temporal interval it was presented. A classical example is the 2-intervals 2-alternatives forced choice task. Tasks that present a reference stimulus in the first interval, and therefore have $n$ intervals and $n-1$ alternatives are also supported \cite<see>[for an example of such tasks]{Grimault2002}.

\subsubsection{Adaptive Interleaved}
This paradigm implements the interleaved adaptive procedure described by \citeA{Jesteadt1980}.

\subsubsection{Weighted Up/Down}
This paradigm implements the weighted up/down adaptive procedure described by \citeA{Kaernbach1991}.

\subsubsection{Weighted Up/Down Interleaved}
This paradigm combines the interleaved adaptive procedure described by \citeA{Jesteadt1980} with the weighted up/down method described by \citeA{Kaernbach1991}.

\subsubsection{Constant m-Intervals n-Alternatives}
This paradigm implements a constant difference method for forced choice tasks with $m$-intervals and $n$-alternatives. For example, it can be used for running a 2-intervals, 2-alternatives forced choice frequency-discrimination task with a constant difference between the stimuli in the standard and comparison intervals.

\subsubsection{Constant 1-Interval 2-Alternatives}
This paradigm implements a constant difference method for tasks with a single observation interval and two response alternatives, such as the ``Yes/No'' signal detection task. 

\subsubsection{Constant 1-Pair Same/Different}
This paradigm implements a constant difference method for ``same/different'' tasks with a single pair of stimuli to compare.

\section{Available Experiments}





%%% Local Variables: 
%%% mode: latex
%%% TeX-master: "pychoacoustics_manual"
%%% End: 

